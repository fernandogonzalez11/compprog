\documentclass{article}
\usepackage[margin=0.5in]{geometry}
\usepackage{listings}
\usepackage{xcolor}
\usepackage{amsmath}
\usepackage{amssymb}

% Define colors for syntax highlighting
\definecolor{codegreen}{rgb}{0,0.6,0}
\definecolor{codegray}{rgb}{0.5,0.5,0.5}
\definecolor{codepurple}{rgb}{0.58,0,0.82}
\definecolor{backcolour}{rgb}{0.95,0.95,0.92}

\lstdefinestyle{cppstyle}{
	backgroundcolor=\color{backcolour},   
	commentstyle=\color{codegreen},
	keywordstyle=\color{magenta},
	numberstyle=\tiny\color{codegray},
	stringstyle=\color{codepurple},
	basicstyle=\ttfamily\footnotesize,
	breakatwhitespace=false,         
	breaklines=true,                 
	captionpos=b,                    
	keepspaces=true,                 
	numbers=left,                    
	numbersep=5pt,                  
	showspaces=false,                
	showstringspaces=false,
	showtabs=false,                  
	tabsize=2,
	language=C++
}

\lstset{style=cppstyle}

\title{}
\author{}
\date{}

\begin{document}
	
	\section*{Segment tree (bueno)}
	\begin{lstlisting}
		/**
		* Segment tree
		* De kactl
		* Indexado en 0, iterativo
		* Actualizacion en punto
		* Consultas de forma [l, r)
		* Complejidad: O(log N)
		*/
		
		struct Mono { /* TODO */
			static Mono zero() { /* TODO */ }
		};
		
		Mono operator+(Mono a, Mono b) { /* TODO */ }
		
		struct Tree {
			vector<Mono> s;
			int n;
			
			Tree(int n) : n(n), s(2*n, Mono::zero()) {}
			
			void update(int pos, T val) {
				for (s[pos += n] = val; pos >>= 1;) s[pos] = s[pos<<1] + s[pos<<1|1];
			}
			
			// [l, r)
			T query(int l, int r) {
				T lv = Mono::zero(), rv = Mono::zero();
				for (l += n, r += n; l < r; l >>= 1, r >>= 1) {
					if (l&1) lv = lv + s[l++];
					if (r&1) rv = s[--r] + rv;
				}
				return lv + rv;
			}
		};
	\end{lstlisting}
	
	\section*{Dijkstra}
	\begin{lstlisting}
		/**
		* Dijkstra
		* Robado de bruno
		* O(mlog(n))
		* Si utilizas esa vara con pesos negativos me enojo
		* Para bfs 0-1, reemplazar cola de prioridad con un deque
		*    Si el peso de la arista es 1, meter al final, sino, al frente
		*/
		
		vector<lld> dijkstra(vector<vector<pair<int, lld>>> &adj, int v) {
			int n = adj.size();
			vector<lld> d(n, LLONG_MAX >> 3);
			d[v] = 0;
			priority_queue<pair<lld, int>> pq;
			pq.emplace(0, v);
			
			while (pq.size()) {
				auto [ndist, u] = pq.top(); pq.pop();
				if (-ndist > d[u]) continue;
				
				for (auto [idx, w] : adj[u]) if (d[idx] > d[u] + w) {
					d[idx] = d[u] + w;
					pq.emplace(-d[idx], idx);
				}
			}
			return d;
		}
	\end{lstlisting}
	
	\section*{Topsort}
	\begin{lstlisting}
		/**
		* Ordenacion topologica
		* Robado de bruno
		* Regresa a una ordenacion topologica de g
		* Si g no es para DAG, devuelve un vector vacio
		* O(n + m)
		*/
		vector<int> topo_sort(int n, vector<vector<int>> &g) {
			vector<int> ret(n, -1), vis(n, 0);
			int pos = n - 1, dag = 1;
			function<void(int)> dfs = [&](int v) {
				vis[v] = 1;
				for (auto u : g[v]) {
					if (vis[u] == 1) dag = 0;
					else if (!vis[u]) dfs(u);
				}
				ret[pos--] = v, vis[v] = 2;
			};
			for (int i = 0; i < n; i++) if (!vis[i]) dfs(i);
			if (!dag) ret.clear();
			return ret;
		}
	\end{lstlisting}
	
	\section*{Puentes y puntos de articulación}
	\begin{lstlisting}
		/**
		* SCC con Tarjan
		* Robado de bruno
		* O(n + m)
		*/
		struct Tarjan {
			vector<vector<int>> g;
			stack<int> s;
			vector<int> vis, comp, id;
			Tarjan(int len, vector<vector<int>> &adj) {
				g = adj;
				vis.assign(len, 0);
				comp.assign(len, 0);
				id.assign(len, 0);
			}
			int dfs(int i, int &t) {
				int lo = id[i] = t++;
				s.push(i);
				vis[i] = 2;
				for (int j : g[i]) {
					if (!vis[j]) lo = min(lo, dfs(j, t));
					else if (vis[j] == 2) lo = min(lo, id[j]);
				}
				if (lo == id[i]) while (1) {
					int u = s.top();
					s.pop();
					vis[u] = 1, comp[u] = i;
					if (u == i) break;
				}
				return lo;
			}
			void tarjan(int n) {
				int t = 0;
				for (int i = 0; i < n; i++) vis[i] = 0;
				for (int i = 0; i < n; i++) if (!vis[i]) dfs(i, t);
			}
		};
	\end{lstlisting}
	
\end{document}